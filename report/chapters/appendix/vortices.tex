\chapter{Vortices}\label{chap:vortices}
		\begin{figure}[H]
			\centering
			\begin{subfigure}[h]{0.45\textwidth}
				\centering
				\includegraphics[width=\textwidth]{../figures/Metropolis/frames/1.pdf}
				\caption{At $T=\num{1.5}$, we are in the high-temperature state where the spins are unordered.}
			\end{subfigure}
			~
			\begin{subfigure}[h]{0.45\textwidth}
				\centering
				\includegraphics[width=\textwidth]{../figures/Metropolis/frames/30.pdf}
				\caption{After cooling the system down to $T=\num{0.02}$ there is one bound vortex pair and the rest of the spins are mostly aligned.}
			\end{subfigure}
			\begin{subfigure}[h]{0.45\textwidth}
				\centering
				\includegraphics[width=\textwidth]{../figures/Metropolis/frames/40.pdf}
				\caption{After some sweeps at $T=\num{0.02}$, the vortex pair has come closer together. The rest of the lattice is uniformly aligned.}
			\end{subfigure}
			~
			\begin{subfigure}[h]{0.45\textwidth}
				\centering
				\includegraphics[width=\textwidth]{../figures/Metropolis/frames/59.pdf}
				\caption{After the vortex-antivortex pair has annihilated, a quasi-stable lattice with ordered spins remains.}
			\end{subfigure}
			\caption[Vortex unbinding of vortex-antivortex pairs at low temperatures]{Simulating the process of vortex unbinding by bringing a thermalized high-temperature unordered lattice slowly to low temperatures as described in~\cref{sec:vortex_unbinding}. Below the critical temperature $T_C$, bound vortex-antivortex pairs appear and slowly annihilate at very low temperatures. The lattice is left in a quasi-ordered low-temperature state.}
			\label{fig:vortex_unbinding}
		\end{figure}