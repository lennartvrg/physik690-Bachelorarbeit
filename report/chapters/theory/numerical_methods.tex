\section{Numerical Methods}\label{sec:theo:numerical_methods}
	As stated in the introduction in~\Cref{chap:introduction}, the goal is to simulate the two-dimensional XY model using the Metropolis-Hastings and the Wolff-Cluster algorithms. In the following sections, we present the algorithms and concepts needed for understanding and implementing the simulation.

	\subsection{Metropolis Algorithm}\label{sec:theo:metropolis}
		The fundamental problem we need to solve is that we want to generate configurations of our system with probability $P(\sigma) = \exp{(-\beta H[\sigma])}/Z$. The problem is that the partition sum $Z$ is not easy to calculate. As shown by~\citet{metropolis}, it is sufficient to calculate the ratio of probabilities instead
		\begin{equation}
			\frac{P(y)}{P(x)} = \exp{(-\beta(H[y] - H[x]))} = \exp{(-\beta \Delta H)}.
		\end{equation}
		This follows from \emph{detailed balance}, which is one of the conditions required for this procedure to work. The condition of \emph{ergodicity} states, that for infinite timescales, all possible configurations must be reached. This procedure generates a Markov chain of dependent configurations that converges to a static distribution. The overall procedure for the lattice is as follows:
		\begin{enumerate}
			\item Calculate $E$, $M$, and $\Upsilon$ for the initial lattice configuration.
			\item Perform a lattice sweep by iterating over all lattice sites. For every site $i$ do:
			\begin{enumerate}
				\item Propose a new angle $\theta_i \in [0,2\pi)$ from a uniform distribution.
				\item Calculate $\Delta H = \Delta E$, $\Delta M$, and $\Delta \Upsilon$ for the proposed new state.
				\item Accept or reject state with probability $P = \min{(1, \exp{(-\beta\Delta H)})}$.
			\end{enumerate}
			\item If the new state is accepted, update the observables $E$, $M$, and $\Upsilon$ by adding $\Delta E$, $\Delta M$, and $\Delta \Upsilon$, respectively.
			\item Add a copy of the observables to the result set and repeat from 2. for a total of $N$ sweeps.
		\end{enumerate}
		
	\subsection{Autocorrelation}
		The configurations obtained from that procedure are dependent on the configurations preceding them.  To measure this correlation, one can introduce the autocorrelation
		\begin{equation}
			C(\Delta t) = \langle(O(t) - \mu_O)(O(t + \Delta t) - \mu_O) \rangle
		\end{equation}
		\cite[eq. 41]{bootstrap} and, to get a size-invariant measure, the normalized autocorrelation function
		\begin{equation}
			\Gamma(\Delta t) = \frac{C(\Delta t)}{C(0)}
		\end{equation}
		\cite[eq. 43]{bootstrap} with $\Delta t$ being the step between samples, $O$ an observable, and $\mu_O$ the mean of that observable. This measure decays exponentially, with the integrated autocorrelation time $\tau$ being the point where samples are no longer correlated
		\begin{equation}
			\tau = \frac{1}{2} + \sum^{T}_{t=1}{\Gamma(t)}
		\end{equation}
		\cite[eq. 46]{bootstrap}. The summation is cut off when it first crosses zero. In the implementation, we use a Discrete Fourier Transform to accelerate this step.
		
	\subsection{Thermalization and Blocking}\label{sec:blocking}
		To obtain independent and identically distributed (iid) samples, we discard the first $\lceil 3\tau \rceil$ values in a thermalization step. We then perform a blocking step, where we divide the remaining samples into chunks of size $\lceil \tau \rceil$ and calculate the mean over each chunk. All further processing will use these blocked samples, which are now iid.
		
		The choice of $\lceil \tau \rceil$ as the blocking stride is based on the lecture notes by~\citet{bootstrap}, but other authors prefer strides of $\lceil 2\tau \rceil$ or $\lceil 3\tau \rceil$. A thorough discussion in \emph{Monte Carlo errors with less errors} by~\citet{monte_carlo_errors} shows that there is an additional factor to the integrated autocorrelation time, which is difficult to account for.
		
	\subsection{Critical Slowing Down}\label{sec:theo:critical_slowing_down}
		Near the critical temperature, the autocorrelation drastically increases, which is due to critical slowing-down effects and the dynamical exponent $z$, which is defined as
		\begin{equation}
			\tau \propto \xi^z
		\end{equation}
		\cite[eq. (6)]{bootstrap}. Here, $\xi$ is the correlation length, which is a measure of how aligned spins are at a distance.
		
		While the correlation length $\xi$ is inherent to our physical system, the dynamical exponent is a property of the algorithm (e.g., Metropolis-Hastings) used to generate our configurations. It is therefore necessary to use other algorithms if we want to lower the dynamical exponent and thus reduce the critical slowing-down effects.
		
	\subsection{Wolff Cluster Algorithm}\label{sec:theo:wolff_cluster}
		As discussed in~\Cref{sec:theo:critical_slowing_down}, the Metropolis-Hastings algorithm is prone to severe critical slowing-down effects. In a desire to mitigate such effects,~\cite{sw} introduced a multi-cluster Monte Carlo algorithm for the Potts spin model, \enquote{giving a highly efficient method of simulation for large systems near criticality} \cite[p. 86]{sw}.
		
		Adjacent to the multi-cluster Swendson and Wang algorithm, the Wollf algorithm exhibits similar improvements in the critical exponent, while utilizing only a single cluster, which makes implementations more straightforward.
		
		The key insight is that the Ising spin-flip operation $\sigma_i \rightarrow -\sigma_i$ needs to be generalized to the $O(2)$ symmetry of the XY model. \cite{wolff} defines the spin-flip as the reflection along the hyperplane orthogonal to an arbitrarily chosen unit vector $\vec{r}$
		\begin{equation}
			R(\vec{r}) \sigma_i = \sigma_i - 2 (\sigma_i \cdot \vec{r}) \vec{r} 
		\end{equation}
		which is an idempotent operation $R(\vec{r})^2=1$ \cite[eq. 3]{wolff}.
	
		The overall procedure for the Wolff algorithm was adapted from~\citet[p. 361]{wolff} with the addition of~\cref{wolf_loop}. The step was added, as the probability of a site joining the lattice is lower for unordered states that emerge at high temperatures. Consequently, the average cluster size tends towards $1$ in those states. This results in a very uneven distribution of potential updates (\emph{marked} sites in the algorithm procedure). At low temperatures, many sites get visited and potentially updated, while at high temperatures, as few as five (1 starting site + 4 neighbours) sites get marked. To mitigate this, we propose~\cref{wolf_loop} as an additional step that ensures a sufficient number of potential updates are made, even at high temperatures.
		
		The overall procedure is the following:
		\begin{enumerate}
			\item Calculate $E$, $M$, and $\Upsilon$ for the initial lattice configuration.
			\item \label{wolf_loop} Perform the following until we have marked $L^2$ lattice sites in total:
			\begin{enumerate}
				\item Choose a random two-dimensional unit vector $\vec{r}$ as the reflection vector.
				\item Select a random lattice site $x$ as the first element of the cluster.
				\item Flip the spin at the initial lattice site $\sigma_x \rightarrow R(\vec{r}) \sigma_x$ and mark site $x$.
				\item \label{wolff_step} Visit all direct unmarked neighbours $y$ of $x$ and add them to the cluster with probability:
					\begin{equation}\label{eq:wolff}
						P(\sigma_x, \sigma_y) = 1 - \exp(\min[0, 2 \beta (\vec{r}\cdot\vec{\sigma_x}) (\vec{r}\cdot\vec{\sigma_y})])
					\end{equation}
					and calculate $\Delta E$, $\Delta M$, and $\Delta \Upsilon$ for the proposed new state.
				\item If site $y$ is accepted into the cluster, its spin is flipped $\sigma_y \rightarrow R(\vec{r}) \sigma_y$.
				\item Site $y$ becomes the new $x$, and we continue from~\cref{wolff_step} until there are no more sites to add to the cluster. The delte observables $\Delta E$, $\Delta M$, and $\Delta \Upsilon$ are updated as the cluster grows.
			\end{enumerate}
			\item Update the observables $E$, $M$, and $\Upsilon$ by adding $\Delta E$, $\Delta M$, and $\Delta \Upsilon$, respectively.
			\item Add a copy of the observables to the result set and continue from~\cref{wolf_loop} until the exit condition is fulfilled.
		\end{enumerate}
		The procedure also fulfills \emph{ergodicity} as there is \enquote{always a nonvanishing probability that [the cluster] $c$ consists of only one site, and that there is always a reflection connecting any two spins} \citep[p. 361]{wolff}.  Further, the condition of~\emph{detailed balance} is also fulfilled~\cite[eq. 6]{wolff}.
	
	\subsection{Bootstrapping}\label{sec:bootstrap}
		Since both algorithms scale with the number of lattice sites, it is computationally impractical to run them for long timescales. Given that we have an observable whose means follow a Gaussian distribution and we already have \enquote{enough} iid samples that cover a sufficient portion of the possible configuration space, we then use bootstrapping to generate additional samples \citep*{bootstrap}.
		\begin{enumerate}
			\item Collect $B$ intermediate means by repeating the following:
			\begin{enumerate}
				\item Take $A$ random samples from the blocked samples obtained in~\cref{sec:blocking} with replacement.
				\item Calculate the mean of those samples and add the result to the set of intermediate means.
			\end{enumerate}
			\item Calculate the sample standard deviation of those $B$ intermediate means as the uncertainty of our observable.
		\end{enumerate}
		The estimate for the observable is still calculated as the mean of the iid samples.