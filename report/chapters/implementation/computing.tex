\subsection{Distributed Computing}\label{sec:impl:computing}
	The temperature scanning of~\Cref{sec:impl:scanning} divides every lattice size $L$ into $N$ steps and $I$ iterations. As the number of configurations far exceeds the core count of a single computer for any realistic choice of $L$, $N$, and $I$, the computational efforts should be scaled out across multiple compute nodes. The implementation addresses the shortcomings mentioned at the beginning of~\Cref{sec:impl} and requires two kinds of nodes.
	
	\subsubsection{Data Node}\label{sec:impl:computing:data}
		The \emph{data} node is responsible for orchestrating the compute nodes and storing the intermediate and final results. The orchestration is realized using a  \emph{PostgreSQL}\footnote{\url{https://www.postgresql.org}} database and does not require any additional software. Having a central \emph{data} node fixes shortcoming~\cref{shortcomings:sqlite} where the nodes need to have access to a common network share.
		
		The schema for the \emph{PostgreSQL} database is presented in~\Cref{chap:schema}. Foreign keys link all relevant database entities, and unique constraints have been added to ensure that every configuration is simulated exactly once.
	
	\subsubsection{Compute Node}\label{sec:impl:computing:compute}
		One or more \emph{compute} nodes are required to run the simulation. These nodes should have modern, high-frequency, multi-core CPUs with \emph{AVX2} support (\cref{sec:impl:optimizations}) as our simulation primarily utilizes the CPU.
		
		The compute nodes only need internal network access to the \emph{data} node to access the \emph{PostgreSQL} instance. To automatically install all required dependencies and compile the program, a \emph{CloudInit}\footnote{\url{https://cloudinit.readthedocs.io/en/latest/index.html}} configuration file is available in the repository\footnote{\url{https://github.com/lennartvrg/physik690-Bachelorarbeit/blob/main/cloud-config.yaml}}. The script also registers the program as a \emph{systemd}\footnote{\url{https://systemd.io/}} service, which provides resilience by restarting the service when it exits with an error code\footnote{\url{https://github.com/lennartvrg/physik690-Bachelorarbeit/blob/main/bachelor.service}}.
