\section{Temperature Scanning}\label{sec:impl:scanning}
	The program splits the temperature interval $T \in (0.0, 3.0]$ into $N$ steps and runs the simulation at each of those temperatures.  As observed by others (\cite{olsson_helicity}), there should be an interesting area near the critical temperature where the magnetic susceptibility peaks.
	
	I therefore choose an iterative approach where the simulation \emph{zooms} into the interesting area near the critical temperature. After having picked the maximum number of iterations $I_\text{max} > 0$ the procedure is as follows:
	\begin{enumerate}
		\item Divide the temperature interval $T \in (0.0, 3.0]$ into $64$ discrete steps\footnote{While the number of steps can be arbritrary, it is sensible to pick a number that evenly divides the number of total cores across all nodes.}.
		\item \label{scanning:iter} Simulate the system at the given on those discrete temperatures using one thread per temperature. An iteration counter $I_\text{c} = 0$ is incremented and, if $I_\text{c} >= I_\text{max}$, the procedure exits.
		\item Find the temperature $T_\text{max}$ where the magnetic susceptibility is maximal.
		\item Pick a new temperature interval $T \in (T_\text{max} - 2 \Delta T,  T_\text{max} + 2 \Delta T)$ where $\Delta T$ is the step size from the last iteration\footnote{As an example: Dividing $T \in (0.0, 3.0]$ into $64$ steps yields $\Delta T = \num{0.046875}$}. The factor of $2$ was chosen to ensure that the next iteration simulates a big enough range around $T_\text{max}$ so that true peak is definitly in that range.
		\item Go to~\cref{scanning:iter}.
	\end{enumerate}
	One downside of this procedure is, that one must be careful to simulate an enough number of sweeps at every temperature value. As the procedure is zooming into the area near the critical temperature, which is also the point where critical slowing down is most severe, the integrated autocorrelation time also increases which in turn means fewer iid samples. This results in bigger statistical errors for the magnetic susceptibility so the procedure might unwillingly choose the wrong $T_\text{max}$ value for the next iteration.