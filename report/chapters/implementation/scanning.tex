\subsection{Temperature Scanning}\label{sec:impl:scanning}
	The program splits the temperature interval $T \in (0.0, 3.0]$ into $N$ steps and simulates each of those temperatures.  As observed by others \citep*{olsson_helicity}, there should be an area near the critical temperature where the magnetic susceptibility peaks.
	
	We therefore chose an iterative approach where the simulation \emph{zooms} in on the area near the critical temperature. After having picked the maximum number of iterations $I_\text{max} > 0$, the procedure is as follows:
	\begin{enumerate}
		\item \label{scanning:init} Divide the temperature interval $T \in (0.0, 3.0]$ into $64$ discrete steps\footnote{While the number of steps can be arbitrary, it is sensible to pick a number that evenly divides the total number of cores across all nodes.}.
		\item \label{scanning:iter} Simulate the system at the given temperatures. An iteration counter $I_\text{c} = 0$ is incremented and, if $I_\text{c} >= I_\text{max}$, the procedure exits.
		\item Find the temperature $T_\text{max}$ where the magnetic susceptibility is maximal.
		\item Pick a new temperature interval $T \in (T_\text{max} - 3 \Delta T,  T_\text{max} + 3 \Delta T)$ where $\Delta T$ is the step size from the last iteration\footnote{As an example: Dividing $T \in (0.0, 3.0]$ into $64$ steps yields $\Delta T = \num{0.046875}$}. The factor of $3$ was chosen to ensure that the next iteration simulates a sufficiently large range around $T_\text{max}$, so that the actual peak is definitely in that range.
		\item Go to~\cref{scanning:iter}.
	\end{enumerate}
	One downside of this procedure is that one must be careful to simulate an adequate number of sweeps at every temperature value. As the procedure \emph{zooms} in on the area near the critical temperature, which is also the point where critical slowing down is most severe, the integrated autocorrelation time increases, meaning there are fewer iid samples. Fewer samples result in larger statistical errors for the magnetic susceptibility, so the procedure might inadvertently choose the wrong $T_\text{max}$ value for the next iteration.