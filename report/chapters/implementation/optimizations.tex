\section{Optimizations}\label{sec:impl:optimizations}
	Apart from distributing the workload across multiple threads and nodes, some optimizations can be made to improve performance on a single thread.
	
	\paragraph{SIMD Instructions}\label{sec:impl:optimizations:simd}
		As introduced in the definitions of the Metropolis (\cref{sec:theo:metropolis}) and Wolff (\cref{sec:theo:wolff_cluster}) algorithms, it is computationally more sensible to calculate the delta energy/magnetization/helicity modulus that a proposed spin update would entail, instead of recalculating the observable for the entire lattice. In our two-dimensional lattice model with nearest neighbour approximation, this means that only the four direct neighbours of a spin $\sigma_i$ contribute to the calculation.
			
		As many modern-day CPUs have the AVX2\footnote{\url{https://edc.intel.com/content/www/us/en/design/ipla/software-development-platforms/client/platforms/alder-lake-desktop/12th-generation-intel-core-processors-datasheet\\-volume-1-of-2/009/intel-advanced-vector-extensions-2-intel-avx2/}} instruction set enabled, which allows us to perform standard mathematical operations on vectors of four $64$-bit scalars simultaneously. We can use this to improve the performance of our implementation. \emph{AVX2} in itself does not provide intrinsics for the operations ($\sin, \cos$) that are needed. Therefore, we will also use the \emph{Intel\textsuperscript{\tiny\textregistered} Short Vector Math Library} (SVML\footnote{\url{https://www.intel.com/content/www/us/en/docs/cpp-compiler/developer-guide-reference/2021-8/intrinsics-for-short-vector-math-library-ops.html}}), which provides methods that generate an optimized sequence of intrinsics for $\sin$ and $\cos$.
		
		The internal \mintinline{c++}{std::vector<double_t>} storing the spins has a row-first ordering in memory. Some of the observables, such as $\lvert M \rvert^2$ and $\Upsilon$, can use this to load four neighbouring spins at once. To facilitate those operations, the internal \mintinline{c++}{std::vector<double_t>} storing the spins is aligned on a $32$-byte boundary.
		
	\paragraph{Compiler Flags}
		The program uses the \mintinline{bash}{-march=native} compiler flag and should therefore be compiled on a system with the same or very similar CPU as the compute nodes. Additionally, the \mintinline{bash}{-ffast-math} flag is used, which sacrifices ISO compliance for execution speed. Although no systematic testing is conducted, the total wall-time improvement on the development machine is $\approx \SI{20}{\percent}$ better, with no observable loss in numerical results. 
	
	\paragraph{Intel Compiler}
		We also want to try the \emph{Intel\textsuperscript{\tiny\textregistered}} oneAPI DPC++/C++ compiler, as the manufacturer claims it offers superiority over \emph{gcc} in HPC workloads. No systematic testing is conducted, but on the development machine (AMD Ryzen 1700), a $\SI{6}{\percent}$ reduction of the total wall-time is observed. Tests on a second machine (\emph{Intel\textsuperscript{\tiny\textregistered} i7-10710U}) yield a $\SI{14}{\percent}$ improvement over $\emph{gcc}$.
			
	


	