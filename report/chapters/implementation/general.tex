\chapter{Implementation}
	To simulate the two dimensional XY model with periodic boundary conditions (\cref{sec:theo:xy_model}) I implemented the numerical methods of~\cref{sec:theo:numerical_methods} in a multi-threaded \emph{C++26} program. A short guide on how to obtain and compile the sources may be found in~\cref{chap:source_code}.
	
	The overall procedure for the simulation is the following:
	\begin{enumerate}
		\item The database is initialized and the initial set of configurations one wants to simulate are inserted into the database. A configuration is defined as the algorithm (Metropolis or Wolff), a lattice size $L$ and a temperature $T$. A detailed description on how the temperature range is scanned follows in~\cref{sec:impl:scanning}.
		\item The program checks whether any existing configuration is incomplete and, if that is the case, executes the following procedure:
			\begin{enumerate}
				\item In the \emph{simulation} stage the worker threads fetch their next configurtion, simulates it for the configured number of sweeps and writes the observables $E$, $E^2$, $M$, $M^2$ and $\Upsilon$ back to the database. The total number of sweeps is split into $n_\text{Chunks}$ Chunks. This is discussed in more details in~\cref{sec:impl:computing}.
				\item In the \emph{bootstrap} stage the worker threads fetch the observables for each configuration simulated beforehand. The observables get bootstrapped and the final results get written to the database.
				\item In the \emph{derivative} phase the workers calculate the observables ($C_V$, $\chi_X$) derived from the bootstrapped results.
			\end{enumerate}
		\item The program simulates 
	\end{enumerate}