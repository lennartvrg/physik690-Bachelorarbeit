\chapter{Implementation}\label{sec:impl}
	To simulate the two-dimensional XY model with periodic boundary conditions (\Cref{sec:theo:xy_model}), we implement the numerical methods of~\Cref{sec:theo:numerical_methods} in a distributed, multi-threaded \emph{C++26} program. A brief guide on how to obtain and compile the sources can be found in~\Cref{chap:source_code}. For the following sections, these two definitions will be helpful:
	\begin{enumerate}
		\item The term \emph{core} refers to physical cores when a compute node has SMT disabled and to threads when SMT is enabled.
		\item The term \emph{configuration} refers to any combination of algorithm, lattice size $L$, and temperature $T$.
	\end{enumerate}
	
	Having worked on the XY model during the \emph{physics760 - Computational Physics} class, one of the shortcomings of the implementation back then was that every compute node simulated exactly one lattice size $L$ from start to finish before saving the final results and moving on to the next lattice size. This procedure is suboptimal for various reasons:
	\begin{enumerate}
		\item \label{shortcomings:cores} The program divides the temperature interval into $N$ steps. When the compute node has more cores than the number of steps, the left-over cores are idle.
		\item Only the final bootstrapped estimates are saved, which is a problem when one wants to increase the number of~\ldots
		\begin{enumerate}
			\item \label{shortcomings:sweeps}  \ldots~sweeps after the simulation finishes. Since the final lattice spins are not saved, the program can not continue the simulation, and the entire simulation has to be rerun from start to finish.
			\item \label{shortcomings:bootstrap} \ldots~bootstrap samples after the simulation finishes. Since the iid observables are not saved, the program cannot redo the bootstrap procedure, and the entire simulation has to be rerun from scratch.
		\end{enumerate}
		\item \label{shortcomings:crashed} When the program or the compute node crashes, only the finished lattice sizes are saved. All lattice sizes that are not yet finished have to start over once the node comes back online.
		\item \label{shortcomings:sqlite} The results are saved on the disk in an SQLite database, which requires that all compute nodes have access to the same network share and that this network share is adequately fast.
	\end{enumerate}