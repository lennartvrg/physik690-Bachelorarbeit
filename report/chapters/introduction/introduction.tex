\chapter{Introduction}\label{chap:introduction}
	In physics, there are a variety of problems for which it is either very hard or impossible to find analytical solutions. These problems often involve many particle systems and/or integrals of high dimensionality. With ever-increasing computing power available to researchers, it is becoming increasingly feasible to study such systems using Monte Carlo methods.
	
	The most well-known application of Monte Carlo methods in physics is the Ising model, a mathematical model commonly used to study phenomena such as phase transitions, critical temperatures, and universality. This model was first introduced and solved in one dimension by~\cite{ising} and later in two dimensions by~\cite{onsager}. It describes the total magnetization of a lattice as the superposition of all spin sites $\sigma_i = \pm 1$. It features a second-order phase transition at a critical temperature $T_C$, where the system transitions from an ordered low-temperature state to an unordered high-temperature state.
	
	One can now generalize the problem from a discrete $\mathbb{Z}_2$ symmetry to a continuous $U(2)$ symmetry. First introduced by~\cite{matsubara} as a model for liquid helium, it is now known as the XY model and it describes the spins as two-dimensional vectors on the unit circle. Unlike the Ising model, the XY model does not exhibit a second-order phase transition. However, as shown by~\cite{kosterlitz} (2016 Nobel Prize in Physics), it does feature something we now call a KT phase transition, where below a critical temperature $T_C$ metastable states can exist. These metastable states are closely related to vortex-antivortex pairs on the lattice, which annihilate as $T_C \rightarrow 0$.
	
	These kinds of systems are often simulated using the Metropolis-Hastings algorithm, which is prone to severe critical slowing-down effects. Cluster update algorithms introduced by~\cite{sw} and~\cite{wolff} reduce critical slowing-down effects and yield more accurate numerical results.
	
	In this bachelor's thesis, we employ Monte Carlo methods to study the two-dimensional XY model. The primary objective is to develop a program that simulates the XY model in a distributed manner, observes critical slowing-down effects when using the Metropolis-Hastings algorithm, and utilizes the Wolff cluster algorithm to mitigate them. The second goal is to observe the physical phenomena of the system, such as the KT phase transition, estimate the critical temperature $T_C$, and identify the annihilation of vortex-antivortex pairs.
	