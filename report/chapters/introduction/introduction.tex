\chapter{Introduction}\label{chap:introduction}
	In physics there are a variaty of problems for which it is either very hard or impossible to find analytical solutions. These problems often involve many particle systems and/or integrals of high dimensionality. With ever increasing compute power available to researchers, it becomes more and more feasible to study such system using Monte Carlo methods.
	
	The perhaps most well-known application of Monte Carlo methods in physics is the Ising model. The Ising model is a mathematical model which is commenly used to study phenomenos like phase transitions, critical temperatures and universality on an --- at least in 1D~\cite{ising} and 2D~\cite{onsager} --- analytically solvable model. It describes the total magnetization of a lattice as the superposition of all spin sites $\sigma_i = \pm 1$ and features a 2nd order phase phase transition at a critical temperature $T_C$ where the system transitions from an ordered low-temperature to a unordered high-temperature state.
	
	One can now generalize the problem from a discrete $\mathbb{Z}_2$ symmetry to a continuous $U(2)$ symmetry. First introduced by~\cite{matsubara}, as a model for liquid helium, it is now known as the XY model and describes the spins as two-dimensional vectors on the unit circle. Unlike the Ising model, the XY model does not have a 2nd order phase transition. However, as shown by~\cite{kosterlitz} (2016 Nobel Prize in Physics), it does feature something we now call a KT phase transition where below a critical temperature $T_C$ metastable states can exist. These metastable states are closely related to vortex/anti-vortex pairs on the lattice which annihilate for $T_C \rightarrow 0$.
	
	These kind of systems are often simulated using the Metropolis-Hastings algorithm which suffers from severe critical slowing down effects. Cluster update algorithms introduced by~\cite{sw} and~\cite{wolff} reduce critical slowing down effects and yield more accurate numerical results.
	
	In this bachelor work I will be employing Monte Carlo methods to study the two dimensional XY model. The first goal is to implement a program that runs the simulation in a distributed manner, observes the critical slowing down effect when using the Metropolis-Hastings algorithm and implements the Wolff cluster algorithm to mitigate these problems. The second goal is to observe the physical phenomenons of the system such as the KT phase transition, estimating the critical temperature $T_C$ and finding vortex/anti-vortex pair annihilation.
	