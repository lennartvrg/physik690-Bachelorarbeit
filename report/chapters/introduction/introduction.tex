\chapter{Introduction}\label{chap:introduction}
	In physics there are a variaty of problems for which it is either very hard or impossible to find analytical solutions. These problems often involve many particle systems and/or integrals of high dimensionality. With ever increasing compute power available to researchers, it becomes more and more feasible to study such system using Monte Carlo methods. Monte Carlo methods is an umbrella term for numerical methods which emply random numbers to simulate a particular system.
	
	The perhaps most well-known application of Monte Carlo methods is the Ising model.  It describes the total magnetization of a lattice as the superposition of
	all spins $\sigma = \pm 1$ on the lattice. The Ising model features a phase transition at a critical temperature where the system transitions from an ordered low-temperature to a unordered high-temperature state.
	
	One can now generalize the problem from a $\mathbb{Z}_2$ symmetry to a continuous $U(2)$ symmetry. This model is called the XY model and describes the spins as two-dimensional vectors on the unit circle. Unlike the Ising model, the XY model does not have a 2nd order phase transition. It does however have something we call KT phase transition where below a critical temperature TC metastable states can exist. These metastable states are closely related to vortex/anti-vortex pairs on the lattice. The discovery of the KT phase transition was awared with the Nobel prize in 2016.
	
	In this bachelor work I will be employing Monte Carlo methods, namely the Metropolis and Wolff algorithms, to simulate the two dimensional XY model. 