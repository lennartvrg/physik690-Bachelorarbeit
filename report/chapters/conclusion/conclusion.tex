\chapter{Conclusion}
	In this work, we implemented a program that simulates the XY model using the Metropolis (\cref{sec:theo:metropolis}) and Wolff (\cref{sec:theo:wolff_cluster}) algorithms. The simulation was able to combine the computational resources of a 16 node cluster effectivly and which is a promising indicator for future bigger runs.
	
	We were able to study the observables energy $E$ (\cref{sec:res:energy:observable}), total absolute magnetization $\lvert M^2 \rvert$ (\cref{sec:res:magnet:observable}), specific heat $C_V$ (\cref{sec:res:cv:observable}) and magnetic susceptibility $\chi$ (\cref{sec:res:xs}). Additionally critical slowing-down effects were observed in~\Cref{sec:res:energy:observable} and, by implementing the Wolff single cluster algorthm, reduced.
	
	In~\Cref{sec:res:temperature} we used the shifted temperature $T^*$~\citep{shifted} to estimate critical temperature $T_C$ for the Metropolis algorithm
	\begin{equation}
		T_{C, \text{Metropolis}} = \SI{0.886(12)}{\J\per\kb}
	\end{equation}
	and then for the Wolff algorithm
	\begin{equation}
		T_{C, \text{Wolff}} = \SI{0.8978(14)}{\J\per\kb}.
	\end{equation}
	The errors were obtained from a bootstrap procedure using the standard deviation $\sigma$. As discussed in~\Cref{sec:res:temperature} our Metropolis estimate is compatible with existing literature values while the Wolff estimate is only compatible in a $4\sigma$ setting. We further note, that the Wolff estimate is more precice by a factor of $\sim 10$.
	
	In~\Cref{sec:vortex_unbinding} we observed the existance and annihilation of vortex-antivortex pairs below the critical temperature $T_C$ when using the Metropolis algorithm. For the Wolff algorithm, the vortex-antivortex pairs dissolve too quickly to visualize.
	
	At last, in~\Cref{sec:scaling}, we discussed the correlation of computational effort and lattice size. We conclude that for bigger lattice sizes and more precise estimates of the critical temperature, further simulations should rely solely on the Wolff algorithm, as it gives more precise estimates in a shorter amount of time. Alternaticly, implementing the Metropolis algorithm for GPU clusters is another promising endeavor as the algorithm can be easily parallelized.