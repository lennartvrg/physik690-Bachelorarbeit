\section{Scaling}\label{sec:scaling}
	\begin{figure}[htbp]
		\centering
		\includegraphics[width=0.8\textwidth]{../figures/Scaling.pdf}
		\caption[Correlation of the computational effort and the lattice size]{Lattice size dependence of the total computation time of the Metropolis and Wolff algorithms}
		\label{fig:scaling}
	\end{figure}
	In~\Cref{fig:schema}, we observe that the total simulation time $T$ is proportional to $L^2$. The Wolff algorithm takes approximately half the time of the Metropolis algorithm, while, as discussed in~\Cref{sec:res:temperature}, coming to a more precise estimate. As the Metropolis algorithm used $24$ chunks and the Wolff algorithm only $6$ chunks, we conclude that the cluster building procedure of the Wolff algorithm is computationally more expensive than a single sweep of the Metropolis procedure.
	
	Note that the simulation ran on a compute cluster in the public cloud, so we cannot account for the effects of noisy neighbours and other external factors. The effects of that can be seen at $L=272$ and $L=288$, where the total simulation time of the Metropolis algorithm is the same for both lattice sizes. The overall trend, however, when looking at all lattice sizes, confirms $T \propto L^2$.