\section{Vortex Unbinding}\label{sec:vortex_unbinding}
	To visualize the vortex unbinding at low temperatures, we performed a single run where we first heated a $64 \times 64$ lattice  to $T = \num{1.5}$ and allowed it to thermalize for $\num{100 000}$ sweeps. The temperature was then lowered back down to $T = \num{0.05}$ in $90$ steps with $20$ sweeps of thermalization at each temperature step. At $T = \num{0.05}$, the system was given $\num{180 000}$ sweeps to allow the vortices to annihilate.
	
	An animation of the procedure using the Metropolis algorithm can be viewed here\footnote{\url{https://www.youtube.com/watch?v=nWonn68vUjc}}, with some key frames shown in~\Cref{fig:vortex_unbinding} of~\Cref{chap:vortices}. Below the critical temperature $T_C$, bound vortex-antivortex pairs appear. As the temperature is further lowered, the pairs come closer together until they annihilate. The lattice is left in a quasi-ordered low-temperature state where the spins mostly align.
	
	Executing the same procedure with the Wolff algorithm yields this\footnote{\url{https://www.youtube.com/watch?v=g_K8_V4TKj4}} animation. Here, one can clearly observe the flipping of big clusters at low temperature. The existance of vortex-antivortex pairs is here more difficult, as the pairs annihialte too quickly. 