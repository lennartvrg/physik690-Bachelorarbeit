\section{Energy per Spin}\label{sec:res:energy}
	\subsection{Observable}\label{sec:res:energy:observable}
		\Observable{Metropolis}{Energy}{energy}
		\Observable{Wolff}{Energy}{energy}
		\Cref{fig:obs:Metropolis:Energy} shows the temperature dependence of the energy per spin (\cref{eq:energy}) using the Metropolis algorithm. We observe the expected quasi-ordered state at low temperatures, where the spins mostly align. The energy tends towards $\SI{-2}{\joule}$ for $T\rightarrow 0$, which matches the Ising model. With increasing temperature, the absolute value of the energy per spin slowly reduces. For $T \rightarrow \infty$ we can expect $E \rightarrow 0$ as the spins become less and less aligned, and the spin energies average out. The errors can be found in~\Cref{sec:errors:energy} of~\Cref{chap:errors}.
		
		Comparing the form of the curve with the results of the Wolff algorithm in~\Cref{fig:obs:Wolff:Energy} shows the same overall form. A slight deviation can be observed for $L = 4$, where at $T = \SI{3}{\joule\per\kb}$, the data point for the Wolff algorithm is slightly lower than that for the Metropolis algorithm. The lower data point is probably due to the small lattice size and the fact that cluster sizes tend towards one at higher temperatures (see~\cref{sec:res:cluster_size}). At these higher temperatures, the acceptance probabilities of both algorithms are smaller, and, as the Wolff algorithm always flips at least one site (the starting site of the cluster-building procedure), we obtain a slightly different curve compared to the Metropolis algorithm, as the Metropolis algorithm may flip as few as zero spins. This effect is amplified by~\cref{wolf_loop} in our Wolff procedure (\cref{sec:theo:wolff_cluster}). During development runs without~\cref{wolf_loop}, it was not possible to even observe the rough form of the curve in the $T \geq T_C$ regimes. The introduction of~\cref{wolf_loop} is therefore a sensible tradeoff.
		
		While the energy per spin curve is visually the same for both algorithms, the residuals and integrated autocorrelation time $\tau$ differ significantly. We plotted the autocorrelation function $\Gamma(\Delta t)$ at the lowest, highest, and the temperature where  the magnetic susceptibility $\chi$ peaked. The autocorrelation function was cut off at the last datapoint where $\Gamma(\Delta t) > \num{0}$.
		
		Comparing the Metropolis algorithm (\cref{fig:obs:Metropolis:Energy:autocorrelation}) and the Wolff algorithm (\cref{fig:obs:Wolff:Energy:autocorrelation}) reveals that low temperatures generally exhibit autocorrelation function curves with long tails. In contrast, high temperatures feature a short autocorrelation curve. Near criticality, the autocorrelation function approaches zero much more quickly when using the Wolff algorithm. This behaviour is expected as the Wolff algorithm is supposed to give \enquote{a highly efficient method of simulation for large systems near criticality}~\cite[p. 86]{sw}.
		\Autocorrelation{Metropolis}{Energy}{energy}
		\Autocorrelation{Wolff}{Energy}{energy}
		
		The advantage becomes even clearer when plotting the integrated autocorrelation time $\tau$ of the Metropolis algorithm (\cref{fig:obs:Metropolis:Energy:integrated}) and the Wolff algorithm (\cref{fig:obs:Wolff:Energy:integrated}). When comparing these graphs, we can see that $\tau$ is much smaller when using the Wolff algorithm. As already discussed in~\Cref{sec:res:cluster_size}, this results in more iid samples and a more accurate and precise measurement of our observables. We conclude that the Wolff algorithm suffers less from critical slowing-down and yields superior estimates of our observables in terms of accuracy and precision.
		\IntegratedAutocorrelation{Metropolis}{Energy}{energy}
		\IntegratedAutocorrelation{Wolff}{Energy}{energy}
	
	\paragraph{Energy Squared per Spin}\label{sec:res:energysquare} A thorough discussion of $E^2$ will be omitted as the general observations of $E$ also hold for $E^2$. The plots are, for completeness' sake, enclosed in~\Cref{sec:observables:energy_squared}, and the errors can be found in~\Cref{sec:errors:energysquare} of~\Cref{chap:errors}..
	
	\subsection{Specific Heat per Spin}\label{sec:res:cv:observable}
		The specific heat capacity $C_V$ can be derived from $\langle E \rangle$ and $\langle E^2 \rangle$  using~\Cref{eq:specific_heat}. Comparing $C_V$ using the Metropolis (\cref{fig:obs:Metropolis:SpecificHeat}) and Wolff (\cref{fig:obs:Wolff:SpecificHeat}) algorithms reveals that both follow the same curve. The specific heat per spin starts small, has a small peak which moves closer to $T_C$ with increasing lattice sizes, and then falls off for $T\rightarrow \infty$. The numeric value, in the $T\rightarrow 0$ limit, is different for each lattice size, but gets smaller with increasing lattice sizes. The errors can be found in~\Cref{sec:errors:specificheat} of~\Cref{chap:errors}.
		
		Near criticality, the curve of $C_V$ obtained using the Wolff algorithm is substantially smoother, which can be attributed to the reduced critical slowing-down effects experienced by the Wolff algorithm.
		\Observable{Metropolis}{SpecificHeat}{specific heat}
		\Observable{Wolff}{SpecificHeat}{specific heat}
	
	\subsection{Helicity Modulus per Spin}\label{sec:res:helicity:observable}
		As introduced in~\Cref{eq:helicity_modulus}, the helicity modulus $\Upsilon$ \enquote{measures the increase in the amount of energy when we rotate the order parameter of a magnetically long-range-order system along a given direction by a small angle} \citep[p. 1]{krueger}. Comparing the curves of Metropolis (\Cref{fig:obs:Metropolis:HelicityModulus}) and Wolff (\Cref{fig:obs:Wolff:HelicityModulus}) reveals no difference in overall form, although the curve obtained using the Wolff algorithm is smoother. For $T \rightarrow 0$, the XY model has a long-range order, and the helicity modulus $\Upsilon$ tends towards $1$. Above $T_C$, the long-range order vanishes and $\Upsilon$ drops towards $0$. The errors can be found in~\Cref{sec:errors:helicity} of~\Cref{chap:errors}.
		\Observable{Metropolis}{HelicityModulus}{helicity modulus}
		\Observable{Wolff}{HelicityModulus}{helicity modulus}
		
		One particular feature of the helicity modulus is that the intersection of the graph of $\frac{2}{\pi} T$ and $\Upsilon$ can be used to estimate $T_C$~\citep{teitel_helicity}. Attempts by~\cite{teitel_helicity} and~\cite{olsson_helicity} show that while this is possible, the results are generally less precise than using the magnetic susceptibility $\chi$ to estimate $T_C$. We will therefore concentrate on the latter in the following sections.

	